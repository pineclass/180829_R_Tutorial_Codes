\documentclass[]{article}
\usepackage{lmodern}
\usepackage{amssymb,amsmath}
\usepackage{ifxetex,ifluatex}
\usepackage{fixltx2e} % provides \textsubscript
\ifnum 0\ifxetex 1\fi\ifluatex 1\fi=0 % if pdftex
  \usepackage[T1]{fontenc}
  \usepackage[utf8]{inputenc}
\else % if luatex or xelatex
  \ifxetex
    \usepackage{mathspec}
  \else
    \usepackage{fontspec}
  \fi
  \defaultfontfeatures{Ligatures=TeX,Scale=MatchLowercase}
\fi
% use upquote if available, for straight quotes in verbatim environments
\IfFileExists{upquote.sty}{\usepackage{upquote}}{}
% use microtype if available
\IfFileExists{microtype.sty}{%
\usepackage{microtype}
\UseMicrotypeSet[protrusion]{basicmath} % disable protrusion for tt fonts
}{}
\usepackage[margin=1in]{geometry}
\usepackage{hyperref}
\hypersetup{unicode=true,
            pdftitle={simple\_reg\_1.R},
            pdfauthor={SANGHOOJEFFREY},
            pdfborder={0 0 0},
            breaklinks=true}
\urlstyle{same}  % don't use monospace font for urls
\usepackage{color}
\usepackage{fancyvrb}
\newcommand{\VerbBar}{|}
\newcommand{\VERB}{\Verb[commandchars=\\\{\}]}
\DefineVerbatimEnvironment{Highlighting}{Verbatim}{commandchars=\\\{\}}
% Add ',fontsize=\small' for more characters per line
\usepackage{framed}
\definecolor{shadecolor}{RGB}{248,248,248}
\newenvironment{Shaded}{\begin{snugshade}}{\end{snugshade}}
\newcommand{\KeywordTok}[1]{\textcolor[rgb]{0.13,0.29,0.53}{\textbf{#1}}}
\newcommand{\DataTypeTok}[1]{\textcolor[rgb]{0.13,0.29,0.53}{#1}}
\newcommand{\DecValTok}[1]{\textcolor[rgb]{0.00,0.00,0.81}{#1}}
\newcommand{\BaseNTok}[1]{\textcolor[rgb]{0.00,0.00,0.81}{#1}}
\newcommand{\FloatTok}[1]{\textcolor[rgb]{0.00,0.00,0.81}{#1}}
\newcommand{\ConstantTok}[1]{\textcolor[rgb]{0.00,0.00,0.00}{#1}}
\newcommand{\CharTok}[1]{\textcolor[rgb]{0.31,0.60,0.02}{#1}}
\newcommand{\SpecialCharTok}[1]{\textcolor[rgb]{0.00,0.00,0.00}{#1}}
\newcommand{\StringTok}[1]{\textcolor[rgb]{0.31,0.60,0.02}{#1}}
\newcommand{\VerbatimStringTok}[1]{\textcolor[rgb]{0.31,0.60,0.02}{#1}}
\newcommand{\SpecialStringTok}[1]{\textcolor[rgb]{0.31,0.60,0.02}{#1}}
\newcommand{\ImportTok}[1]{#1}
\newcommand{\CommentTok}[1]{\textcolor[rgb]{0.56,0.35,0.01}{\textit{#1}}}
\newcommand{\DocumentationTok}[1]{\textcolor[rgb]{0.56,0.35,0.01}{\textbf{\textit{#1}}}}
\newcommand{\AnnotationTok}[1]{\textcolor[rgb]{0.56,0.35,0.01}{\textbf{\textit{#1}}}}
\newcommand{\CommentVarTok}[1]{\textcolor[rgb]{0.56,0.35,0.01}{\textbf{\textit{#1}}}}
\newcommand{\OtherTok}[1]{\textcolor[rgb]{0.56,0.35,0.01}{#1}}
\newcommand{\FunctionTok}[1]{\textcolor[rgb]{0.00,0.00,0.00}{#1}}
\newcommand{\VariableTok}[1]{\textcolor[rgb]{0.00,0.00,0.00}{#1}}
\newcommand{\ControlFlowTok}[1]{\textcolor[rgb]{0.13,0.29,0.53}{\textbf{#1}}}
\newcommand{\OperatorTok}[1]{\textcolor[rgb]{0.81,0.36,0.00}{\textbf{#1}}}
\newcommand{\BuiltInTok}[1]{#1}
\newcommand{\ExtensionTok}[1]{#1}
\newcommand{\PreprocessorTok}[1]{\textcolor[rgb]{0.56,0.35,0.01}{\textit{#1}}}
\newcommand{\AttributeTok}[1]{\textcolor[rgb]{0.77,0.63,0.00}{#1}}
\newcommand{\RegionMarkerTok}[1]{#1}
\newcommand{\InformationTok}[1]{\textcolor[rgb]{0.56,0.35,0.01}{\textbf{\textit{#1}}}}
\newcommand{\WarningTok}[1]{\textcolor[rgb]{0.56,0.35,0.01}{\textbf{\textit{#1}}}}
\newcommand{\AlertTok}[1]{\textcolor[rgb]{0.94,0.16,0.16}{#1}}
\newcommand{\ErrorTok}[1]{\textcolor[rgb]{0.64,0.00,0.00}{\textbf{#1}}}
\newcommand{\NormalTok}[1]{#1}
\usepackage{graphicx,grffile}
\makeatletter
\def\maxwidth{\ifdim\Gin@nat@width>\linewidth\linewidth\else\Gin@nat@width\fi}
\def\maxheight{\ifdim\Gin@nat@height>\textheight\textheight\else\Gin@nat@height\fi}
\makeatother
% Scale images if necessary, so that they will not overflow the page
% margins by default, and it is still possible to overwrite the defaults
% using explicit options in \includegraphics[width, height, ...]{}
\setkeys{Gin}{width=\maxwidth,height=\maxheight,keepaspectratio}
\IfFileExists{parskip.sty}{%
\usepackage{parskip}
}{% else
\setlength{\parindent}{0pt}
\setlength{\parskip}{6pt plus 2pt minus 1pt}
}
\setlength{\emergencystretch}{3em}  % prevent overfull lines
\providecommand{\tightlist}{%
  \setlength{\itemsep}{0pt}\setlength{\parskip}{0pt}}
\setcounter{secnumdepth}{0}
% Redefines (sub)paragraphs to behave more like sections
\ifx\paragraph\undefined\else
\let\oldparagraph\paragraph
\renewcommand{\paragraph}[1]{\oldparagraph{#1}\mbox{}}
\fi
\ifx\subparagraph\undefined\else
\let\oldsubparagraph\subparagraph
\renewcommand{\subparagraph}[1]{\oldsubparagraph{#1}\mbox{}}
\fi

%%% Use protect on footnotes to avoid problems with footnotes in titles
\let\rmarkdownfootnote\footnote%
\def\footnote{\protect\rmarkdownfootnote}

%%% Change title format to be more compact
\usepackage{titling}

% Create subtitle command for use in maketitle
\newcommand{\subtitle}[1]{
  \posttitle{
    \begin{center}\large#1\end{center}
    }
}

\setlength{\droptitle}{-2em}

  \title{simple\_reg\_1.R}
    \pretitle{\vspace{\droptitle}\centering\huge}
  \posttitle{\par}
    \author{SANGHOOJEFFREY}
    \preauthor{\centering\large\emph}
  \postauthor{\par}
      \predate{\centering\large\emph}
  \postdate{\par}
    \date{Tue Jun 26 19:38:34 2018}


\begin{document}
\maketitle

\begin{Shaded}
\begin{Highlighting}[]
\CommentTok{# 회귀분석 기초와 상관분석}

\CommentTok{# 회귀분석의 이해}
\CommentTok{# http://www.shodor.org/interactivate/activities/Regression/}

\CommentTok{# 회귀분석 기초1(Simple regression analysis 1)}
\CommentTok{# 가장 간단한 형태의 회귀분석은 한 개의 설명변수와 한 개의 반응변수 간의 관계식을 찾는 문제}
\CommentTok{# 이러한 경우를 단순회귀분석이라고 한다. }

\CommentTok{# 단순회귀분석을 위해선 가장 먼저 산점도를 통해 설명변수와 반응변수 간 선형 관계가 있는지 확인해야한다.}
\CommentTok{# 산점도는 plot()함수를 이용}

\KeywordTok{data}\NormalTok{(cars) }\CommentTok{# 속도에 따른 정지거리(dist)에 관한 데이터}

\NormalTok{cars}
\end{Highlighting}
\end{Shaded}

\begin{verbatim}
##    speed dist
## 1      4    2
## 2      4   10
## 3      7    4
## 4      7   22
## 5      8   16
## 6      9   10
## 7     10   18
## 8     10   26
## 9     10   34
## 10    11   17
## 11    11   28
## 12    12   14
## 13    12   20
## 14    12   24
## 15    12   28
## 16    13   26
## 17    13   34
## 18    13   34
## 19    13   46
## 20    14   26
## 21    14   36
## 22    14   60
## 23    14   80
## 24    15   20
## 25    15   26
## 26    15   54
## 27    16   32
## 28    16   40
## 29    17   32
## 30    17   40
## 31    17   50
## 32    18   42
## 33    18   56
## 34    18   76
## 35    18   84
## 36    19   36
## 37    19   46
## 38    19   68
## 39    20   32
## 40    20   48
## 41    20   52
## 42    20   56
## 43    20   64
## 44    22   66
## 45    23   54
## 46    24   70
## 47    24   92
## 48    24   93
## 49    24  120
## 50    25   85
\end{verbatim}

\begin{Shaded}
\begin{Highlighting}[]
\KeywordTok{plot}\NormalTok{(cars}\OperatorTok{$}\NormalTok{speed, cars}\OperatorTok{$}\NormalTok{dist)}

\CommentTok{# 산점도를 그려보면 속도(speed)와 정지거리(dist) 간 선형관계가 있어보인다.}
\CommentTok{# 설명변수와 반응변수 간 상관정도를 정량적으로 확인하기 위해 상관분석을 실시한다.}
\CommentTok{# R에서는 cor()함수를 이용}

\KeywordTok{cor}\NormalTok{(cars}\OperatorTok{$}\NormalTok{speed, cars}\OperatorTok{$}\NormalTok{dist)}
\end{Highlighting}
\end{Shaded}

\begin{verbatim}
## [1] 0.8068949
\end{verbatim}

\begin{Shaded}
\begin{Highlighting}[]
\KeywordTok{cor.test}\NormalTok{(cars}\OperatorTok{$}\NormalTok{speed, cars}\OperatorTok{$}\NormalTok{dist) }\CommentTok{# 가설검정까지 원하는 경우}
\end{Highlighting}
\end{Shaded}

\begin{verbatim}
## 
##  Pearson's product-moment correlation
## 
## data:  cars$speed and cars$dist
## t = 9.464, df = 48, p-value = 1.49e-12
## alternative hypothesis: true correlation is not equal to 0
## 95 percent confidence interval:
##  0.6816422 0.8862036
## sample estimates:
##       cor 
## 0.8068949
\end{verbatim}

\begin{Shaded}
\begin{Highlighting}[]
\CommentTok{# 그러면 최적 선형식은 어떻게 구할까? (모수추정)}
\CommentTok{# 최적 선형식을 위해선 dist = beta_0 + beta_1 * speed의 beta_0와 beta_1을 추정해야 한다. }

\CommentTok{# 만약 beta_0= -2, beta_1= 4 를 넣는다면}
\CommentTok{# dist = -2 + 4* speed로 예측된다. }

\NormalTok{hat.dist =}\StringTok{ }\OperatorTok{-}\DecValTok{2} \OperatorTok{+}\StringTok{ }\FloatTok{3.8} \OperatorTok{*}\StringTok{ }\NormalTok{cars}\OperatorTok{$}\NormalTok{speed}

\CommentTok{# 실제값과 예측값의 차이를 만들어보자. }
\NormalTok{diff =}\StringTok{ }\NormalTok{cars}\OperatorTok{$}\NormalTok{dist }\OperatorTok{-}\StringTok{ }\NormalTok{hat.dist}

\CommentTok{# 실제값과 예측값이 차이의 제곱 합을 계산하면 다음과 같다. }
\KeywordTok{sum}\NormalTok{(diff}\OperatorTok{^}\DecValTok{2}\NormalTok{)  }
\end{Highlighting}
\end{Shaded}

\begin{verbatim}
## [1] 20544.12
\end{verbatim}

\begin{Shaded}
\begin{Highlighting}[]
\CommentTok{# 그러면 위의 값을 최소로 하는 식이 최적식이 아닐까? (최소제곱법)}
\NormalTok{f.out <-}\StringTok{ }\ControlFlowTok{function}\NormalTok{(x) \{}
\NormalTok{  hat.dist =}\StringTok{ }\NormalTok{x[}\DecValTok{1}\NormalTok{] }\OperatorTok{+}\StringTok{ }\NormalTok{x[}\DecValTok{2}\NormalTok{] }\OperatorTok{*}\StringTok{ }\NormalTok{cars}\OperatorTok{$}\NormalTok{speed}
\NormalTok{  diff =}\StringTok{ }\NormalTok{cars}\OperatorTok{$}\NormalTok{dist }\OperatorTok{-}\StringTok{ }\NormalTok{hat.dist}
  \KeywordTok{return}\NormalTok{(}\KeywordTok{sum}\NormalTok{(diff}\OperatorTok{^}\DecValTok{2}\NormalTok{))}
\NormalTok{\}}

\KeywordTok{f.out}\NormalTok{(}\DataTypeTok{x=}\KeywordTok{c}\NormalTok{(}\OperatorTok{-}\DecValTok{2}\NormalTok{,}\DecValTok{4}\NormalTok{))}
\end{Highlighting}
\end{Shaded}

\begin{verbatim}
## [1] 25171
\end{verbatim}

\begin{Shaded}
\begin{Highlighting}[]
\KeywordTok{f.out}\NormalTok{(}\DataTypeTok{x=}\KeywordTok{c}\NormalTok{(}\OperatorTok{-}\DecValTok{17}\NormalTok{,}\DecValTok{4}\NormalTok{))}
\end{Highlighting}
\end{Shaded}

\begin{verbatim}
## [1] 11491
\end{verbatim}

\begin{Shaded}
\begin{Highlighting}[]
\KeywordTok{f.out}\NormalTok{(}\DataTypeTok{x=}\KeywordTok{c}\NormalTok{(}\OperatorTok{-}\DecValTok{18}\NormalTok{,}\DecValTok{4}\NormalTok{))}
\end{Highlighting}
\end{Shaded}

\begin{verbatim}
## [1] 11379
\end{verbatim}

\begin{Shaded}
\begin{Highlighting}[]
\KeywordTok{f.out}\NormalTok{(}\DataTypeTok{x=}\KeywordTok{c}\NormalTok{(}\OperatorTok{-}\DecValTok{19}\NormalTok{,}\DecValTok{4}\NormalTok{))}
\end{Highlighting}
\end{Shaded}

\begin{verbatim}
## [1] 11367
\end{verbatim}

\begin{Shaded}
\begin{Highlighting}[]
\KeywordTok{f.out}\NormalTok{(}\DataTypeTok{x=}\KeywordTok{c}\NormalTok{(}\OperatorTok{-}\DecValTok{20}\NormalTok{,}\DecValTok{4}\NormalTok{))}
\end{Highlighting}
\end{Shaded}

\begin{verbatim}
## [1] 11455
\end{verbatim}

\begin{Shaded}
\begin{Highlighting}[]
\KeywordTok{f.out}\NormalTok{(}\DataTypeTok{x=}\KeywordTok{c}\NormalTok{(}\OperatorTok{-}\DecValTok{21}\NormalTok{,}\DecValTok{4}\NormalTok{))}
\end{Highlighting}
\end{Shaded}

\begin{verbatim}
## [1] 11643
\end{verbatim}

\begin{Shaded}
\begin{Highlighting}[]
\CommentTok{# 대충 최적값을 짐작하면 beta_0 = -19, beta_1 =4 이다. }
\CommentTok{# 하지만 이 값이 최적점일까? NO}

\KeywordTok{optim}\NormalTok{(}\KeywordTok{c}\NormalTok{(}\DecValTok{0}\NormalTok{,}\DecValTok{1}\NormalTok{), f.out)}\OperatorTok{$}\NormalTok{par }\CommentTok{# 수치적 방법을 이용하여 최적값을 찾을 수 있다.}
\end{Highlighting}
\end{Shaded}

\begin{verbatim}
## [1] -17.578151   3.932216
\end{verbatim}

\begin{Shaded}
\begin{Highlighting}[]
\CommentTok{# 이러한 모수 추정방법을 최소제곱법이라 한다. }
\CommentTok{# 하지만 위의 방법으로 수치적 방법을 이용하지 않더라도 lm()함수를 이용하면 결과 확인 가능}

\NormalTok{out <-}\StringTok{ }\KeywordTok{lm}\NormalTok{(cars}\OperatorTok{$}\NormalTok{dist}\OperatorTok{~}\NormalTok{cars}\OperatorTok{$}\NormalTok{speed) }\CommentTok{# 반응변수 ~ 설명변수 식을 lm()에 넣어주면 된다.}
\NormalTok{out <-}\StringTok{ }\KeywordTok{lm}\NormalTok{(dist}\OperatorTok{~}\NormalTok{speed, }\DataTypeTok{data=}\NormalTok{cars) }\CommentTok{# 또 다른 표현 }

\KeywordTok{summary}\NormalTok{(out)  }\CommentTok{# 결과 확인}
\end{Highlighting}
\end{Shaded}

\begin{verbatim}
## 
## Call:
## lm(formula = dist ~ speed, data = cars)
## 
## Residuals:
##     Min      1Q  Median      3Q     Max 
## -29.069  -9.525  -2.272   9.215  43.201 
## 
## Coefficients:
##             Estimate Std. Error t value Pr(>|t|)    
## (Intercept) -17.5791     6.7584  -2.601   0.0123 *  
## speed         3.9324     0.4155   9.464 1.49e-12 ***
## ---
## Signif. codes:  0 '***' 0.001 '**' 0.01 '*' 0.05 '.' 0.1 ' ' 1
## 
## Residual standard error: 15.38 on 48 degrees of freedom
## Multiple R-squared:  0.6511, Adjusted R-squared:  0.6438 
## F-statistic: 89.57 on 1 and 48 DF,  p-value: 1.49e-12
\end{verbatim}

\begin{Shaded}
\begin{Highlighting}[]
\CommentTok{# 모수의 추정방법으로 우도함수를 이용한 최대우도법이 있으나, 모수추정치는 동일하여 설명은 생략한다.}

\CommentTok{# 모형평가}
\NormalTok{out <-}\StringTok{ }\KeywordTok{lm}\NormalTok{(dist}\OperatorTok{~}\NormalTok{speed, }\DataTypeTok{data=}\NormalTok{cars) }\CommentTok{# 또 다른 표현 }
\KeywordTok{summary}\NormalTok{(out)}
\end{Highlighting}
\end{Shaded}

\begin{verbatim}
## 
## Call:
## lm(formula = dist ~ speed, data = cars)
## 
## Residuals:
##     Min      1Q  Median      3Q     Max 
## -29.069  -9.525  -2.272   9.215  43.201 
## 
## Coefficients:
##             Estimate Std. Error t value Pr(>|t|)    
## (Intercept) -17.5791     6.7584  -2.601   0.0123 *  
## speed         3.9324     0.4155   9.464 1.49e-12 ***
## ---
## Signif. codes:  0 '***' 0.001 '**' 0.01 '*' 0.05 '.' 0.1 ' ' 1
## 
## Residual standard error: 15.38 on 48 degrees of freedom
## Multiple R-squared:  0.6511, Adjusted R-squared:  0.6438 
## F-statistic: 89.57 on 1 and 48 DF,  p-value: 1.49e-12
\end{verbatim}

\begin{Shaded}
\begin{Highlighting}[]
\CommentTok{# summary() 함수의 가장 처음에는 함수식을 나타낸다.}
\CommentTok{# Residuals 부분은 실제 데이터에서 관측된 잔차를 보여준다.}
\CommentTok{# Residual =  관측값 - 예측값}

\NormalTok{obs <-}\StringTok{ }\NormalTok{cars}\OperatorTok{$}\NormalTok{dist }\CommentTok{# 실제 정지거리}
\NormalTok{pred <-}\StringTok{ }\OperatorTok{-}\FloatTok{17.5791}\OperatorTok{+}\FloatTok{3.9324}\OperatorTok{*}\NormalTok{cars}\OperatorTok{$}\NormalTok{speed }\CommentTok{# 추정된 선형식을 통한 예측값}
\NormalTok{resd <-}\StringTok{ }\NormalTok{obs}\OperatorTok{-}\StringTok{ }\NormalTok{pred}

\KeywordTok{summary}\NormalTok{(resd) }\CommentTok{# summary(out)의 Residuals값과 동일}
\end{Highlighting}
\end{Shaded}

\begin{verbatim}
##      Min.   1st Qu.    Median      Mean   3rd Qu.      Max. 
## -29.06890  -9.52520  -2.27170   0.00014   9.21490  43.20150
\end{verbatim}

\begin{Shaded}
\begin{Highlighting}[]
\NormalTok{out}\OperatorTok{$}\NormalTok{residuals }\CommentTok{# 잔차를 불러오기}
\end{Highlighting}
\end{Shaded}

\begin{verbatim}
##          1          2          3          4          5          6 
##   3.849460  11.849460  -5.947766  12.052234   2.119825  -7.812584 
##          7          8          9         10         11         12 
##  -3.744993   4.255007  12.255007  -8.677401   2.322599 -15.609810 
##         13         14         15         16         17         18 
##  -9.609810  -5.609810  -1.609810  -7.542219   0.457781   0.457781 
##         19         20         21         22         23         24 
##  12.457781 -11.474628  -1.474628  22.525372  42.525372 -21.407036 
##         25         26         27         28         29         30 
## -15.407036  12.592964 -13.339445  -5.339445 -17.271854  -9.271854 
##         31         32         33         34         35         36 
##   0.728146 -11.204263   2.795737  22.795737  30.795737 -21.136672 
##         37         38         39         40         41         42 
## -11.136672  10.863328 -29.069080 -13.069080  -9.069080  -5.069080 
##         43         44         45         46         47         48 
##   2.930920  -2.933898 -18.866307  -6.798715  15.201285  16.201285 
##         49         50 
##  43.201285   4.268876
\end{verbatim}

\begin{Shaded}
\begin{Highlighting}[]
\KeywordTok{residuals}\NormalTok{(out)}
\end{Highlighting}
\end{Shaded}

\begin{verbatim}
##          1          2          3          4          5          6 
##   3.849460  11.849460  -5.947766  12.052234   2.119825  -7.812584 
##          7          8          9         10         11         12 
##  -3.744993   4.255007  12.255007  -8.677401   2.322599 -15.609810 
##         13         14         15         16         17         18 
##  -9.609810  -5.609810  -1.609810  -7.542219   0.457781   0.457781 
##         19         20         21         22         23         24 
##  12.457781 -11.474628  -1.474628  22.525372  42.525372 -21.407036 
##         25         26         27         28         29         30 
## -15.407036  12.592964 -13.339445  -5.339445 -17.271854  -9.271854 
##         31         32         33         34         35         36 
##   0.728146 -11.204263   2.795737  22.795737  30.795737 -21.136672 
##         37         38         39         40         41         42 
## -11.136672  10.863328 -29.069080 -13.069080  -9.069080  -5.069080 
##         43         44         45         46         47         48 
##   2.930920  -2.933898 -18.866307  -6.798715  15.201285  16.201285 
##         49         50 
##  43.201285   4.268876
\end{verbatim}

\begin{Shaded}
\begin{Highlighting}[]
\NormalTok{out}\OperatorTok{$}\NormalTok{fitted.values }\CommentTok{# 추정된 선형식으로 예측된 값을 선형식을 세우지 않고 불러올 수 있다.}
\end{Highlighting}
\end{Shaded}

\begin{verbatim}
##         1         2         3         4         5         6         7 
## -1.849460 -1.849460  9.947766  9.947766 13.880175 17.812584 21.744993 
##         8         9        10        11        12        13        14 
## 21.744993 21.744993 25.677401 25.677401 29.609810 29.609810 29.609810 
##        15        16        17        18        19        20        21 
## 29.609810 33.542219 33.542219 33.542219 33.542219 37.474628 37.474628 
##        22        23        24        25        26        27        28 
## 37.474628 37.474628 41.407036 41.407036 41.407036 45.339445 45.339445 
##        29        30        31        32        33        34        35 
## 49.271854 49.271854 49.271854 53.204263 53.204263 53.204263 53.204263 
##        36        37        38        39        40        41        42 
## 57.136672 57.136672 57.136672 61.069080 61.069080 61.069080 61.069080 
##        43        44        45        46        47        48        49 
## 61.069080 68.933898 72.866307 76.798715 76.798715 76.798715 76.798715 
##        50 
## 80.731124
\end{verbatim}

\begin{Shaded}
\begin{Highlighting}[]
\KeywordTok{fitted}\NormalTok{(out) }
\end{Highlighting}
\end{Shaded}

\begin{verbatim}
##         1         2         3         4         5         6         7 
## -1.849460 -1.849460  9.947766  9.947766 13.880175 17.812584 21.744993 
##         8         9        10        11        12        13        14 
## 21.744993 21.744993 25.677401 25.677401 29.609810 29.609810 29.609810 
##        15        16        17        18        19        20        21 
## 29.609810 33.542219 33.542219 33.542219 33.542219 37.474628 37.474628 
##        22        23        24        25        26        27        28 
## 37.474628 37.474628 41.407036 41.407036 41.407036 45.339445 45.339445 
##        29        30        31        32        33        34        35 
## 49.271854 49.271854 49.271854 53.204263 53.204263 53.204263 53.204263 
##        36        37        38        39        40        41        42 
## 57.136672 57.136672 57.136672 61.069080 61.069080 61.069080 61.069080 
##        43        44        45        46        47        48        49 
## 61.069080 68.933898 72.866307 76.798715 76.798715 76.798715 76.798715 
##        50 
## 80.731124
\end{verbatim}

\begin{Shaded}
\begin{Highlighting}[]
\CommentTok{# Coefficients 에서는 회귀모형의 계수와 이 계수의 통계적 유의성을 보여준다.}
\CommentTok{# Estimate 열은 절편과 계수의 추정치}
\CommentTok{# dist = -17.5791+3.9324*speed}
\CommentTok{# Pr(>|t|)는 t 분포를 이용하여 각 변수가 유의한지 판단. 기준은 일반적으로 0.05}
\CommentTok{# 만약, p-value가 0.05보다 크면 혜당 계수가 0이라는 귀무가설을 기각할 수 없으므로 0으로 봐야한다.}


\CommentTok{# 마지막으로 결정계수 (Multiple R-squared) 와  회귀모형의 유의성을 의미하는 F통계량이 제시됨}
\CommentTok{# 여기서 결정계수란?? 선형모형의 설명력으로 해석 }

\KeywordTok{var}\NormalTok{(cars}\OperatorTok{$}\NormalTok{dist) }\CommentTok{# Var(관측값)}
\end{Highlighting}
\end{Shaded}

\begin{verbatim}
## [1] 664.0608
\end{verbatim}

\begin{Shaded}
\begin{Highlighting}[]
\KeywordTok{var}\NormalTok{(cars}\OperatorTok{$}\NormalTok{dist}\OperatorTok{-}\KeywordTok{fitted}\NormalTok{(out)) }\OperatorTok{+}\StringTok{ }\KeywordTok{var}\NormalTok{(}\KeywordTok{fitted}\NormalTok{(out)) }\CommentTok{# Var(관측값-예측값)+Var(예측값)}
\end{Highlighting}
\end{Shaded}

\begin{verbatim}
## [1] 664.0608
\end{verbatim}

\begin{Shaded}
\begin{Highlighting}[]
\NormalTok{SST =}\StringTok{ }\KeywordTok{sum}\NormalTok{((cars}\OperatorTok{$}\NormalTok{dist }\OperatorTok{-}\StringTok{ }\KeywordTok{mean}\NormalTok{(cars}\OperatorTok{$}\NormalTok{dist))}\OperatorTok{^}\DecValTok{2}\NormalTok{)}
\NormalTok{SSE =}\StringTok{ }\KeywordTok{sum}\NormalTok{((cars}\OperatorTok{$}\NormalTok{dist }\OperatorTok{-}\StringTok{ }\KeywordTok{fitted}\NormalTok{(out))}\OperatorTok{^}\DecValTok{2}\NormalTok{)}
\NormalTok{SSR =}\StringTok{ }\KeywordTok{sum}\NormalTok{((}\KeywordTok{fitted}\NormalTok{(out)}\OperatorTok{-}\KeywordTok{mean}\NormalTok{(cars}\OperatorTok{$}\NormalTok{dist))}\OperatorTok{^}\DecValTok{2}\NormalTok{)}
\NormalTok{SST }\OperatorTok{==}\StringTok{ }\NormalTok{SSE}\OperatorTok{+}\StringTok{ }\NormalTok{SSR  }\CommentTok{# 논리 확인}
\end{Highlighting}
\end{Shaded}

\begin{verbatim}
## [1] FALSE
\end{verbatim}

\begin{Shaded}
\begin{Highlighting}[]
\CommentTok{# 모형이 잘 맞는다는건 관측값과 예측값이 비슷하다고 볼 수 있다. 즉 SSE가 0에 가까워짐}
\CommentTok{# SSR/SST는 전제분산 중 예측값으로 설명되는 분산의 비}
\CommentTok{# Multiple R-squared로 의미는 반응변수의 분산 중 설명변수로 설명되는 분산의 비율}

\NormalTok{SSR}\OperatorTok{/}\NormalTok{SST }
\end{Highlighting}
\end{Shaded}

\begin{verbatim}
## [1] 0.6510794
\end{verbatim}

\begin{Shaded}
\begin{Highlighting}[]
\CommentTok{# R^2 는 0과 1범위에 존재하며 단순회귀모형의 경우 상관계수의 제곱과 같다.}
\KeywordTok{cor}\NormalTok{(cars}\OperatorTok{$}\NormalTok{speed, cars}\OperatorTok{$}\NormalTok{dist)}\OperatorTok{^}\DecValTok{2}
\end{Highlighting}
\end{Shaded}

\begin{verbatim}
## [1] 0.6510794
\end{verbatim}

\begin{Shaded}
\begin{Highlighting}[]
\CommentTok{# F 통계량은 full model : dist = beta_0 + beta_1 * speed}
\CommentTok{#            Reduced model : dist = beta_0 }
\CommentTok{# 간 차이를 비교한 값. 즉 통계적으로 유의미한다는건 설명변수가 반응변수에 영향을 미침}

\NormalTok{model1 <-}\StringTok{ }\KeywordTok{lm}\NormalTok{(dist}\OperatorTok{~}\NormalTok{speed, }\DataTypeTok{data=}\NormalTok{cars)}
\NormalTok{model2 <-}\StringTok{ }\KeywordTok{lm}\NormalTok{(dist}\OperatorTok{~}\DecValTok{1}\NormalTok{, }\DataTypeTok{data=}\NormalTok{cars)}
\KeywordTok{anova}\NormalTok{(model1, model2)}
\end{Highlighting}
\end{Shaded}

\begin{verbatim}
## Analysis of Variance Table
## 
## Model 1: dist ~ speed
## Model 2: dist ~ 1
##   Res.Df   RSS Df Sum of Sq      F   Pr(>F)    
## 1     48 11354                                 
## 2     49 32539 -1    -21186 89.567 1.49e-12 ***
## ---
## Signif. codes:  0 '***' 0.001 '**' 0.01 '*' 0.05 '.' 0.1 ' ' 1
\end{verbatim}

\begin{Shaded}
\begin{Highlighting}[]
\CommentTok{# 즉 speed가 유의미한 설명변수이다. }

\KeywordTok{anova}\NormalTok{(out) }\CommentTok{#회귀모형에서의 분산분석결과}
\end{Highlighting}
\end{Shaded}

\begin{verbatim}
## Analysis of Variance Table
## 
## Response: dist
##           Df Sum Sq Mean Sq F value   Pr(>F)    
## speed      1  21186 21185.5  89.567 1.49e-12 ***
## Residuals 48  11354   236.5                     
## ---
## Signif. codes:  0 '***' 0.001 '**' 0.01 '*' 0.05 '.' 0.1 ' ' 1
\end{verbatim}

\begin{Shaded}
\begin{Highlighting}[]
\CommentTok{# 새로운 값이 있을 때 예측은? predict()함수 이용}
\NormalTok{out <-}\StringTok{ }\KeywordTok{lm}\NormalTok{(dist}\OperatorTok{~}\NormalTok{speed, }\DataTypeTok{data=}\NormalTok{cars)}
\KeywordTok{predict}\NormalTok{(out, }\DataTypeTok{newdata=}\KeywordTok{data.frame}\NormalTok{(}\DataTypeTok{speed=}\KeywordTok{c}\NormalTok{(}\DecValTok{3}\NormalTok{,}\DecValTok{4}\NormalTok{,}\DecValTok{5}\NormalTok{)))}
\end{Highlighting}
\end{Shaded}

\begin{verbatim}
##         1         2         3 
## -5.781869 -1.849460  2.082949
\end{verbatim}

\begin{Shaded}
\begin{Highlighting}[]
\KeywordTok{predict}\NormalTok{(out, }\DataTypeTok{newdata=}\KeywordTok{data.frame}\NormalTok{(}\DataTypeTok{speed=}\KeywordTok{c}\NormalTok{(}\DecValTok{3}\NormalTok{,}\DecValTok{4}\NormalTok{,}\DecValTok{5}\NormalTok{)), }\DataTypeTok{interval=}\StringTok{"confidence"}\NormalTok{) }
\end{Highlighting}
\end{Shaded}

\begin{verbatim}
##         fit       lwr       upr
## 1 -5.781869 -17.02659  5.462853
## 2 -1.849460 -12.32954  8.630624
## 3  2.082949  -7.64415 11.810048
\end{verbatim}

\begin{Shaded}
\begin{Highlighting}[]
\CommentTok{# 신뢰구간의 하한과 상한 제시, 평균적인 차량에 대한 신뢰구간}

\KeywordTok{predict}\NormalTok{(out, }\DataTypeTok{newdata=}\KeywordTok{data.frame}\NormalTok{(}\DataTypeTok{speed=}\KeywordTok{c}\NormalTok{(}\DecValTok{3}\NormalTok{,}\DecValTok{4}\NormalTok{,}\DecValTok{5}\NormalTok{)), }\DataTypeTok{interval=}\StringTok{"prediction"}\NormalTok{) }
\end{Highlighting}
\end{Shaded}

\begin{verbatim}
##         fit       lwr      upr
## 1 -5.781869 -38.68565 27.12192
## 2 -1.849460 -34.49984 30.80092
## 3  2.082949 -30.33359 34.49948
\end{verbatim}

\begin{Shaded}
\begin{Highlighting}[]
\CommentTok{# 특정 속도를 가잔 차량 한대의 제동거리는 평균적인 차량에 비해 오차가 크므로 범위가 더 넓어짐}

\CommentTok{# 단순회귀모형의 시각화}
\KeywordTok{plot}\NormalTok{(cars}\OperatorTok{$}\NormalTok{speed, cars}\OperatorTok{$}\NormalTok{dist)}
\KeywordTok{abline}\NormalTok{(}\KeywordTok{coef}\NormalTok{(out), }\DataTypeTok{col=}\StringTok{"blue"}\NormalTok{)}
\end{Highlighting}
\end{Shaded}

\includegraphics{simple_reg_1_files/figure-latex/unnamed-chunk-1-1.pdf}

\begin{Shaded}
\begin{Highlighting}[]
\NormalTok{speed <-}\StringTok{ }\KeywordTok{seq}\NormalTok{(}\KeywordTok{min}\NormalTok{(cars}\OperatorTok{$}\NormalTok{speed), }\KeywordTok{max}\NormalTok{(cars}\OperatorTok{$}\NormalTok{speed),.}\DecValTok{1}\NormalTok{)}
\NormalTok{pred.dist <-}\StringTok{ }\KeywordTok{predict}\NormalTok{(out, }\DataTypeTok{newdata=}\KeywordTok{data.frame}\NormalTok{(}\DataTypeTok{speed=}\NormalTok{speed), }\DataTypeTok{interval=}\StringTok{"confidence"}\NormalTok{)}
\KeywordTok{matplot}\NormalTok{(speed, pred.dist, }\DataTypeTok{type=}\StringTok{'n'}\NormalTok{)}
\KeywordTok{matlines}\NormalTok{(speed, pred.dist, }\DataTypeTok{lty=}\KeywordTok{c}\NormalTok{(}\DecValTok{1}\NormalTok{,}\DecValTok{2}\NormalTok{,}\DecValTok{2}\NormalTok{), }\DataTypeTok{col=}\DecValTok{1}\NormalTok{) }\CommentTok{# 선형 회귀식은 직선, 신뢰구간은 점선으로 표현}
\KeywordTok{matpoints}\NormalTok{(cars}\OperatorTok{$}\NormalTok{speed, cars}\OperatorTok{$}\NormalTok{dist, }\DataTypeTok{pch=}\DecValTok{1}\NormalTok{)}
\end{Highlighting}
\end{Shaded}

\includegraphics{simple_reg_1_files/figure-latex/unnamed-chunk-1-2.pdf}

\begin{Shaded}
\begin{Highlighting}[]
\NormalTok{###########################################################}
\CommentTok{# 단순회귀분석 Summary 1}

\CommentTok{# 1. 설명변수와 반응변수의 산점도를 그린다. plot()}
\CommentTok{#    - 설명변수와 반응변수 간 1차 선형관계가 있는지 확인한다.}

\CommentTok{# 2. 상관분석을 통해 설명변수와 반응변수의 1차 선형관계를 확인한다. cor()}
\CommentTok{#    - p-value<0.05 이하 이면 유의미한 관계가 있다고 판단.}

\CommentTok{# 3. lm()함수를 이용하여 1차 선형식을 추정한다.}

\CommentTok{# 4. F통계량으로 설명변수의 회귀모형의 유의성을 확인한다.}

\CommentTok{# 5. 결정계수를 통해 선형회귀모형의 설명력을 정량적으로 계산한다.}

\CommentTok{# 6. 추정된 회귀계수를 통해 선형식을 구한다.}

\CommentTok{# 7. predict()함수로 새로운 값의 예측값을 계산한다. }

\NormalTok{alligator =}\StringTok{ }\KeywordTok{data.frame}\NormalTok{(}
  \DataTypeTok{lnLength =} \KeywordTok{c}\NormalTok{(}\FloatTok{3.87}\NormalTok{, }\FloatTok{3.61}\NormalTok{, }\FloatTok{4.33}\NormalTok{, }\FloatTok{3.43}\NormalTok{, }\FloatTok{3.81}\NormalTok{, }\FloatTok{3.83}\NormalTok{, }\FloatTok{3.46}\NormalTok{, }\FloatTok{3.76}\NormalTok{,}
               \FloatTok{3.50}\NormalTok{, }\FloatTok{3.58}\NormalTok{, }\FloatTok{4.19}\NormalTok{, }\FloatTok{3.78}\NormalTok{, }\FloatTok{3.71}\NormalTok{, }\FloatTok{3.73}\NormalTok{, }\FloatTok{3.78}\NormalTok{),}
  \DataTypeTok{lnWeight =} \KeywordTok{c}\NormalTok{(}\FloatTok{4.87}\NormalTok{, }\FloatTok{3.93}\NormalTok{, }\FloatTok{6.46}\NormalTok{, }\FloatTok{3.33}\NormalTok{, }\FloatTok{4.38}\NormalTok{, }\FloatTok{4.70}\NormalTok{, }\FloatTok{3.50}\NormalTok{, }\FloatTok{4.50}\NormalTok{,}
               \FloatTok{3.58}\NormalTok{, }\FloatTok{3.64}\NormalTok{, }\FloatTok{5.90}\NormalTok{, }\FloatTok{4.43}\NormalTok{, }\FloatTok{4.38}\NormalTok{, }\FloatTok{4.42}\NormalTok{, }\FloatTok{4.25}\NormalTok{)}
\NormalTok{)}

\CommentTok{# Q.1 다음은 악어의 길이와 무게로 구성된 자료이다.}
\CommentTok{#      연구자가 악어의 길이로 무게를 예측하기 위한 선형식을 구한다고 한다. }
\CommentTok{#      적합한 선형식은?}
\CommentTok{# Q.2. summary()함수를 이용하여 회귀분석 결과에 대해 해석하세요.}
\CommentTok{# Q.3 길이가 4.5인 악어가 잡혔다고 한다. 이 악어의 예상 무게는? }
\CommentTok{# Q.4 회귀분석 결과를 시각화 하세요 }

\KeywordTok{plot}\NormalTok{(alligator}\OperatorTok{$}\NormalTok{lnWeight, alligator}\OperatorTok{$}\NormalTok{lnLength) }\CommentTok{# 산점도}
\end{Highlighting}
\end{Shaded}

\includegraphics{simple_reg_1_files/figure-latex/unnamed-chunk-1-3.pdf}


\end{document}
